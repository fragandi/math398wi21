\documentclass{article}
\usepackage[utf8]{inputenc}


%%%%%%%%%%%%%%%%%%%%%%%%%%%%%%%%%%%%%%%%%%%%%%%%%%%%%%%%%%%%%%%%%%%%%%%%%%%%%%

% Redefine margins and other page formatting



%packages I need



\usepackage{tikz}
\usepackage{tikz-cd}


\usetikzlibrary{arrows,decorations.pathmorphing,decorations.pathreplacing,positioning,shapes.geometric,shapes.misc,decorations.markings,decorations.fractals,calc,patterns}

\usepackage{graphics}
\usepackage[margin=1in]{geometry}
\usepackage[all]{xy}
\usepackage{amsrefs}
\usepackage{amsmath}
\usepackage{amsthm}
\usepackage{amssymb}
\usepackage{amsfonts}
\usepackage{amsxtra}     % Use various AMS packages


\usepackage{epsfig}
\usepackage{verbatim}
\usepackage{enumitem}
\usepackage{algorithmicx}
\usepackage{algpseudocode}

%%%%%%%%%%%%%%%%%%%%%%%%%%%%%%%%%%%%%%%%%%%%%%%%%%%%%%%%%%%%%%%%%%%%%%%%%%%%%%%

% Various theorem environments. All of the following have the same numbering
% system as theorem.

\theoremstyle{plain}
\newtheorem{Theorem}{Theorem}
\newtheorem{Proposition}[Theorem]{Proposition}
\newtheorem{Corollary}[Theorem]{Corollary}
\newtheorem{Lemma}[Theorem]{Lemma}
\newtheorem{Question}[Theorem]{Question}
\newtheorem{Conjecture}[Theorem]{Conjecture}
\newtheorem{Assumption}[Theorem]{Assumption}
\newtheorem{Algorithm}[Theorem]{Algorithm}

\theoremstyle{definition}
\newtheorem{Definition}[Theorem]{Definition}
\newtheorem{Notation}[Theorem]{Notation}
\newtheorem{Condition}[Theorem]{Condition}
\newtheorem{Example}[Theorem]{Example}
\newtheorem{Introduction}[Theorem]{Introduction}

\theoremstyle{remark}
\newtheorem{Remark}[Theorem]{Remark}
\include{header}
%%%%%%%%%%%%%%%%%%%%%%%%%%%%%%%%%%%%%%%%%%%%%%%%%%%%%%%%%%%%%%%%%%%%%%%%%%%%%%%

% \numberwithin{Theorem}{chapter}        % Numbers theorems "x.y" where x
                                        % is the section number, y is the
                                        % theorem number

% \renewcommand{\thechapter}{\arabic{chapter}}
  
% \renewcommand{\thetheorem}{\roman{chapter}.\arabic{theorem}}

%\makeatletter                          % This sequence of commands will
%\let\c@equation\c@theorem              % incorporate equation numbering
%\makeatother                           % into the theorem numbering scheme

%\renewcommand{\theenumi}{(\roman{enumi})}

%%%%%%%%%%%%%%%%%%%%%%%%%%%%%%%%%%%%%%%%%%%%%%%%%%%%%%%%%%%%%%%%%%%%%%%%%%%%%

%%%%%%%%%%%%%%%%%%%%%%%%%%%%%%%%%%%%%%%%%%%%%%%%%%%%%%%%%%%%%%%%%%%%%%%%%%%%%%%%
%%%%%%%%%%%%%%%%%%%%%%%%%%%%%%%%%%%%%%%%%%%%%%%%%%%%%%%%%%%%%%%%%%%%%%%%%%%%%%%

% When you first define a new word, use this macro to make it stand out
% EG We say that an abelian group $I$ is \newword{injective} if, for any
% injection $G \to H$, and any map $G \to I$, there is a map $H \to I$ making the
% obvious diagram commute.

\newcommand{\newword}[1]{\textbf{\emph{#1}}}

%Arrows
\newcommand{\into}{\hookrightarrow}
\newcommand{\onto}{\twoheadrightarrow}

%Things LaTeX names by appearance, rather than meaning
% By now, I've learned the standard LaTeX names, but I remember they used to give me trouble, so here are some macros
%You can create your own macros with newcommand

\newcommand{\isom}{\cong} %The isomorphism symbol
\newcommand{\union}{\cup}
\newcommand{\intersection}{\cap}
\newcommand{\bigunion}{\bigcup}
\newcommand{\bigintersection}{\bigcap}
\newcommand{\disjointunion}{\sqcup}
\newcommand{\bigdisjointunion}{\bigsqcup}


%with DeclareMathOperator you can define a new command which in math environment looks like a mathematical function

%Some multiletter functions
\DeclareMathOperator{\Hom}{Hom}
\DeclareMathOperator{\Ext}{Ext}
\DeclareMathOperator{\End}{End}
\DeclareMathOperator{\Tor}{Tor}
\DeclareMathOperator{\Ker}{Ker}
\DeclareMathOperator{\CoKer}{CoKer}
\DeclareMathOperator{\Spec}{Spec}
\DeclareMathOperator{\MaxSpec}{MaxSpec}
\DeclareMathOperator{\Proj}{Proj}
\DeclareMathOperator{\Frac}{Frac}
\renewcommand{\Im}{\mathop{\mathrm{Im}}}


\DeclareMathOperator{\Obj}{Obj}
\DeclareMathOperator{\kernel}{ker}
\DeclareMathOperator{\im}{im}
\DeclareMathOperator{\spa}{span}
\DeclareMathOperator{\init}{in}
\DeclareMathOperator{\rk}{rk}


%Their calligraphic versions; use these for the sheaf constructions
\DeclareMathOperator{\HHom}{\mathcal{H}\emph{om}}
\DeclareMathOperator{\EExt}{\mathcal{E}\emph{xt}}
\DeclareMathOperator{\EEnd}{\mathcal{E}\emph{nd}}
\DeclareMathOperator{\TTor}{\mathcal{T}\emph{or}}
\DeclareMathOperator{\KKer}{\mathcal{K}\emph{er}}
\DeclareMathOperator{\CCoKer}{\mathcal{C}\emph{o}\mathcal{K}\emph{er}}
\newcommand{\IIm}{\mathop{\mathcal{I}\emph{m}}}
\DeclareMathOperator{\codim}{codim}



%This makes alternating tensors look right in displayed equations
\newcommand{\Alt}{\bigwedge\nolimits}


%Blackboard bold letters.
\renewcommand{\AA}{\mathbb{A}}
\newcommand{\BB}{\mathbb{B}}
\newcommand{\CC}{\mathbb{C}}
\newcommand{\DD}{\mathbb{D}}
\newcommand{\EE}{\mathbb{E}}
\newcommand{\FF}{\mathbb{F}}
\newcommand{\GG}{\mathbb{G}}
\newcommand{\HH}{\mathbb{H}}
\newcommand{\II}{\mathbb{I}}
\newcommand{\JJ}{\mathbb{J}}
\newcommand{\KK}{\mathbb{K}}
\newcommand{\LL}{\mathbb{L}}
\newcommand{\MM}{\mathbb{M}}
\newcommand{\NN}{\mathbb{N}}
\newcommand{\OO}{\mathbb{O}}
\newcommand{\PP}{\mathbb{P}}
\newcommand{\QQ}{\mathbb{Q}}
\newcommand{\RR}{\mathbb{R}}
\renewcommand{\SS}{\mathbb{S}}
\newcommand{\TT}{\mathbb{T}}
\newcommand{\UU}{\mathbb{U}}
\newcommand{\VV}{\mathbb{V}}
\newcommand{\WW}{\mathbb{W}}
\newcommand{\XX}{\mathbb{X}}
\newcommand{\YY}{\mathbb{Y}}
\newcommand{\ZZ}{\mathbb{Z}}

%Calligraphic letters

\newcommand{\cA}{\mathcal{A}}
\newcommand{\cB}{\mathcal{B}}
\newcommand{\cC}{\mathcal{C}}
\newcommand{\cD}{\mathcal{D}}
\newcommand{\cE}{\mathcal{E}}
\newcommand{\cF}{\mathcal{F}}
\newcommand{\cG}{\mathcal{G}}
\newcommand{\cH}{\mathcal{H}}
\newcommand{\cI}{\mathcal{I}}
\newcommand{\cJ}{\mathcal{J}}
\newcommand{\cK}{\mathcal{K}}
\newcommand{\cL}{\mathcal{L}}
\newcommand{\cM}{\mathcal{M}}
\newcommand{\cN}{\mathcal{N}}
\newcommand{\cO}{\mathcal{O}}
\newcommand{\cP}{\mathcal{P}}
\newcommand{\cQ}{\mathcal{Q}}
\newcommand{\cR}{\mathcal{R}}
\newcommand{\cS}{\mathcal{S}}
\newcommand{\cT}{\mathcal{T}}
\newcommand{\cU}{\mathcal{U}}
\newcommand{\cV}{\mathcal{V}}
\newcommand{\cW}{\mathcal{W}}
\newcommand{\cX}{\mathcal{X}}
\newcommand{\cY}{\mathcal{Y}}
\newcommand{\cZ}{\mathcal{Z}}

% I used a macro for Id. If one writes $Id$ the "I" and "d" are far apart, because Latex interprets it a product of I and d. 
\DeclareMathOperator{\Id}{Id}

% some other random commands 

\newcommand{\plu}{Pl\"ucker }
\newcommand{\gro}{Gr\"obner }

\newcommand{\N}{\mathbb{N}^n_0}
\newcommand{\G}{G= \{g_1, \ldots, g_t \}}
\newcommand{\p}{\partial}

\newcommand{\sub}{\subset}
\newcommand{\cont}{\supset}


\newcommand{\R}{\KK[x_1, \ldots, x_n]}
\newcommand{\E}{\we(x_1, \ldots, x_n)}
\newcommand{\x}{\mathbf{x}}
\newcommand{\xa}{\x^{\mathbf{a}}}
\newcommand{\y}{\mathbf{y}}
\newcommand{\ya}{\y^{\mathbf{a}}}
\newcommand{\w}{\mathbf{w}}
\newcommand{\bfu}{\mathbf{u}}
\newcommand{\bfa}{\mathbf{a}}
\newcommand{\bfb}{\mathbf{b}}
\newcommand{\bfc}{\mathbf{c}}
\newcommand{\bfd}{\mathbf{d}}
\newcommand{\bfz}{\mathbf{0}}

%I can never spell lambda
\newcommand{\la}{\lambda}


\title{Automatic Geometric Theorem Proving}
\author{Sam Ratliff}
\date{19 March 2021}
\begin{document}

\maketitle



\section{Automatic Geometric Theorem Proving with Gröbner Bases}
\subsection{Irreducible Varieties}

\section{Automatic Geometric Theorem Proving with Wu's Method}
\subsection{Pseudodivision}
Pseudodivision behaves similar to one variable polynomial division. Let $f$ and $g$ be polynomials in the field $k[x_1,\dots,x_n,y]$.
First, we select a variable $y$ to pseudodivide with respect to. Then, we rewrite $f$ and $g$ in the form 
\begin {align*}
    f & = c_py^p + \cdots + c_1y + c_0\\
    g & = d_my^m + \cdots + d_1y + d_0
\end{align*}
This allows us to isolate $y$ in each monomial. It is important to note that in normal one variable polynomial division, the coefficients of $y$ are in the field $k$. 
In polynomial pseudodivision, however, the coefficients of $y$ are polynomials in $k[x_1,\dots,x_n]$.

We also need to add an additional condition to single variable polynomial division for pseudodivision to be useful. 
To ensure that the pseudodivision works, we multiply $f$ by a power of $d_m$. 
We do this to ensure that the leading term of $d_m f$, the polynomial that we are dividing, is divisible by the leading term of $g$, the polynomial that we are dividing by. 
In other words, we do not know if $c_p$ is divisible by $d_m$, but we do know that there exists some $s \geq 0$ such that $d_m^s\cdot c_p$ is divisible by $d_m$.
While $s = 1$ will easily satisfy this relationship for the first leading terms of $f$ and $g$, we may need to pick a larger $s$ for terms later in the polynomial.

We can also think about the multiple of $d_m^s$ in another way. 
Instead of multiplying by a power of $d_m$ before dividing, we use one variable polynomial division with respect ot $y$ to divide $f$ by $g$ and we don't worry about whether $c_p$ is divisible by $d_m$. 
This means that some of the coefficients of our pseudoquotient and pseudoremainder may end up being rational polynomials. 
To rectify this, we multiply both sides by the smallest power of $d_m$ that will cancel the denominators that still remain in the pseudoquotient and pseudoremainder.
\begin{Proposition}[Pseudodivision]
Let $f, g \in k[x_1,\dots,x_n,y]$, where $deg(g,y) \leq deg(f,y)$. Then, in order to isolate $y$, rewrite $f$ and $g$ as follows:
\begin {align*}
    f & = c_py^p + \cdots + c_1y + c_0\\
    g & = d_my^m + \cdots + d_1y + d_0
\end{align*}
Then, there exists an equation of the form $$d_m^s f = qg + r,$$ where $q,r \in k[x_1,\dots,x_n,y]$, $s \geq 0$, and either $r = 0$ or $deg(r,y) < m$.
\end{Proposition}

We denote the pseudoremainder $r$ of $f$ pseudodivided by $g$ with respect to $y$ as $Rem(f,g,y)$.
One nice property of the pseudoremainder is that, as stated in the propositoin above, $deg(r,y) < deg(f,y) = m$. 
Thus, if we replace $f$ with $r$ and repeatedly pseudodivide, we can eliminate the variable $y$ from an equation without losing other important information about it. 
This operation is similar to mapping a number to its equivalence class modulo $n$. 
In the case of integer equivalence classes, we are only interested in the remainder, which functionally behaves the same as the original integer under the operations of the ring.

\subsection{Triangularization}
Let $f_1',\dots,f_n' \in k[u_1,\dots,u_m,x_1,\dots,x_n]$ be the hypotheses derived from our geometric system.
By using pseudodivision, we can reduce these hypotheses down to triangular form. To do this, we take the $f_i'$ that has the lowest nonzero degree in $x_n$ and swap it with $f_n'$.
Then, for each $f_j'$ where $j < n$, if $f_j'$ depends on $x_n$, we replace $f_j'$ with $Rem(f_j', f_n, x_n)$. We repeat this process until $f_n'$ is the only hypothesis that depends on $x_n$. 
Then, we find the $f_i'$ that has the lowest nonzero degree in $x_{n-1}$ and swap it with $f_{n-1}'$.
We repeat this process until we have reached a set of hypotheses of the form
\begin{align*}
    f_1' & = f_1'(u_1,\dots,u_m,x_1)\\
    f_2' & = f_2'(u_1,\dots,u_m,x_2)\\
    \vdots & \\
    f_n' & = f_n'(u_n,\dots,u_m,x_1,\dots,x_n)
\end{align*}

\subsection{Successive Pseudodivision}

A quick note about notation: unfortunately, the book uses $f$ and $g$ to introduce pseudodivision. We divide $f$ by $g$ in the definition of pseudodivision.
It also uses the set of $f_1,\dots,f_n$ to represent the triangularized polynomials and uses $g$ to represent the conclusion of our proof.
This means that we end up dividing $g$ by the set of $f_1,\dots,f_n$ in this section, which is the opposite of what we had before. 
This tripped me up on my first reading, so I wanted to clarify that point here. Anyway, back to the math.

Much like in the triangularization procedure, we can use pseudodivision on any polynomial in order to eliminate the variables $x_1,\dots,x_n$ from its remainder.
We do this with the following process.

Let $R_n = g$. Then, compute $R_{i-1} = Rem(R_i,f_i,x_i)$. Each successive pseudodivision will eliminate each $x_i$. 
Additionally, we know that each $x_i$ will not be reintroduced in a later pseudodivision because the system has been triangularized.
Thus, when we reach $R_0$, all of the $x_i$s will have been eliminated and we will have a polynomial that depends only on $u_1,\dots,u_m$.


\subsection{Wu's Method}
Wu's method lets us use pseudodivision to determine whether a conclusion $g$ follows from a triangularized set of hypotheses $f_1,\dots,f_n$.
Once again, however, Wu's method is not guaranteed to work on reducible varieties. As such, decomposing a variety into its irreducible components and determining $V'$ may still be required.
\begin{Theorem}[Wu's Method]
    Let $R_0$ be the final remainder computed by the successive pseudodivision of $g$ by the set of triangularized hypotheses $f_1,\dots,f_n$ as discussed above.
    Let $d_i$ be the leading coefficient of each $f_i$. 
    Then, there are nonnegative integers $s_1,\dots,s_n$ and polynomials $A_1,\dots,A_n$ in the ring $\mathbb{R}[u_1,\dots,u_m,x_1\dots,x_n]$ such that
    $$d_{1}^{s_1}\cdot d_2^{s_2} \cdots d_n^{s_n} g = A_1 f_1 + \cdots + A_n f_n + R_0.$$
    Additionally, if $R_0$ is the zero polynomial, then $g$ is zero at every point of $V' \setminus \textbf{V}(d_1d_2\cdots d_n) \subseteq \mathbb{R}^{m+n}$.
\end{Theorem}
The proof of the first part of this theorem is rather intuitive. 
By rearranging the definition of pseudodivision, we can see that after our first step of successive pseudodivision, we can see that $$R_{n-1} = d_n^{s_n} g - q_nf_n.$$
When we pseudodivide again, we get
\begin{align*}
    R_{n-2} & = d_{n-1}^{s_{n-1}}(d_n^{s_n} g - q_nf_n) - q_{n-1}f_{n-1}\\
            & = d_{n-1}^{s_{n-1}}d_n^{s_n} g - q_{n-1}f_{n-1} - d_{n-1}^{s_{n-1}}q_nf_n \\
\end{align*}
From here, it is apparent that the $d_i^{s_i}$ terms will accumulate in front of $g$ and each $f_i$ will also be multiplied by some polynomial.

If $R_0$ is the zero polynomial, then $$d_{1}^{s_1} \cdots d_n^{s_n} g = A_1 f_1 + \cdots + A_n f_n.$$
Therefore, for every point on the variety $\textbf{V}(f_1,\dots,f_n)$, each $f_i = 0$, so at least one polynomial on the left side of the equation must equal zero. 

We know that since $V'$ is an irreducible component of $\textbf{V}(f_1,\dots,f_n)$, it must hold that $V' \subseteq \textbf{V}(f_1,\dots,f_n)$.
Once again, for all points on $V'$, each $f_i = 0$, so it also holds on the points in $V'$ that at least one polynomial on the left side of the equation must equal zero.  
Thus, by removing the points where any $d_i = 0$ from our variety, we find that $g = 0$ on $V' \setminus \textbf{V}(d_1d_2\cdots d_n)$. 

I wrote a basic implementation of the pseudodivision algorithm in the file \texttt{6.5.6.m2} and used it to prove Example 1 from Section 6.4. 
I used this implementation to follow Wu's method for triangularizing the hypotheses and testing the conclusions of the parallelogram proof.
I found that once I had set up my pseudodivision algorithm, the process went very smoothly and was much less time consuming to complete than the original method presented by the book in 6.4.

Wu's method and its variants are generally recognized as the computationally preferable automatic geometric theorem prover and are used much more often than the Gröbner basis method.
This is because the triangularization and pseudodivision algorithms are significantly less computationally demanding than determining the Gröbner basis of an ideal.
The \gro basis of an ideal preserves many more nice properties of the ideal than the triangularization procedure. 
It is unsurprising, then, that computing a \gro basis is a more intricate and time-consuming process than triangularizing the same set of functions.
Since the added benefits of using a \gro basis are not used in automatic geometric theorem proving, it is more computationally efficient to use a solver like Wu's method.

\end{document}
